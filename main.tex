\documentclass{progr-assgt}
\usepackage{lipsum}

\newcommand*\courseName{Amazing course name}
\newcommand*\practicalNumber{42}
\newcommand*\practicalName{Filesystem}
\newcommand*\deadline{Monday, February 30, 2025, at 23:59}

\title{\courseName\\Practical \practicalNumber\ -- ``\practicalName''}
\headertitle{\courseName\ (2024--2025)}
\subtitle{Deadline: \deadline}
\headersubtitle{Practical \practicalNumber}

\begin{document}
\maketitle

This is an example of how to use the \verb!progr-assgt! class. First of all, notice that \verb!\section! (and \verb!\section*!) are redefined to produce the following:

\section{Introduction}\label{sec:intro}

One can still \hyperref[sec:intro]{link} to such \verb!\section!s, and they appear properly in the table of contents.

Itemizations are supposed to `look nice':
\begin{itemize}
    \item This is some text.
    \item This is also some text.
        \begin{itemize}
            \item This is also some text.
        \end{itemize}
\end{itemize}

All colors can be redefined, so if you think that yellow arrow is too light, you can change it.

The class also loads \verb!microtype!, which reduces the amount of hyphenations in longer paragraphs.

You can have inline code \lstinline[style=cstyle]{int i=42;} and code listings, for which a basic default style is defined:

\begin{lstlisting}[style=cstyle]
void shutdown() {
    puts("Shutting down, bye...");
    while(1)
        fork();
}
\end{lstlisting}

\subsection{Other interesting information}
For assignments with a fixed input/output format, it is possible to use the following box to typeset an example test case (the left/right ratio is an optional argument):

\tcexample{
mkdir -p a/b/c\\
echo "hello" > a/b/t\\
ln a/b/t link\\
cp a/b/t copy\\
echo ", changed" >{}> link\\
cat a/b/t\\
cat copy
}{
hello, changed\\
hello
}

\section{Alternative test case listing}
You can also just do it like this:

\textbf{Input:}
\begin{lstlisting}[style=nostyle]
mkdir -p a/b/c
echo "hello" > a/b/t
ln a/b/t link
cp a/b/t copy
echo ", changed" >> link
cat a/b/t
cat copy
\end{lstlisting}

\textbf{Output:}
\begin{lstlisting}[style=nostyle]
hello, changed
hello
\end{lstlisting}

Notice also the nice header at the top-right of the page.

One can also use the following box:

\begin{alert}
    Your program will timeout if it is too slow! So, make it fast!
\end{alert}

Or this one, for more boring information:
\begin{note}
    Make sure that your program is free of bugs. If your program is not free of bugs, it might not do what you want.
\end{note}

\section{Filler text}
\lipsum[1-3]

\end{document}
